\documentclass{ltr} %{{{1 vim: set ts=2 sw=2 tw=100 fdm=marker aw spell:
  \def\PRENOM{Prénom}
     \def\NOM{Nom}
     \def\RUE{Rue 0}
     \def\NPA{0000}
   \def\VILLE{Ville}
   \def\NATEL{000 000 00 00}
\def\COURRIEL{adresse@invalide.net}
        \lieu{Lieu}
        \date{Date du timbre postal}

\begin{document}%{{{1
    \institut{institut}
\newcommand\corpdelettre[1]{%
\conc{Révocation contribution FEDE #1}

\opening{Madame,\\ Monsieur,}

Me sentant totalement lâché,
	voyant mes conditions de travail se dégrader,
	ma charge et la pénibilité augmenter,
	alors que mon salaire baisse,
	je ne vois qu'une solution,
	en découdre avec le conseil d'état.
Comme beaucoup de collègues,
	j'avais posé de gros espoirs sur le combat commencé en 2013
	tout en sachant pertinemment que cela prendrait du temps,
	et qu'il ne suffirait pas de se voir une fois dans une ambiance bon enfant pour que les choses bougent.

La FEDE a baissé les bras en acceptant que tous soit remis sur le tapis chaque année,
	sans aucune prise en compte des sacrifices,
	nombreux,
	que la fonction publique a déjà acceptés.
Par la suite,
	la FEDE ne s'est pas donnée la peine de renégocier pour 2015.
Elle a donc préféré faire le jeu des politiciens plutôt que de défendre le personnel de l'état.
Je ne vois pas en quoi cette FEDE me représente.
	
En conséquence,
	je vous demande de ne plus verser ma contribution,
	pourtant symbolique,
	à cette institution.

\closing{%
	En espérant vous compter parmi les personnes voulant un service public digne de ce nom,
	je vous adresse,
	Madame,
	Monsieur,
	mes salutations les meilleures.
}

}% fin du corp

%{{{1 Lettre FEDE
\begin{letter}{\adret FEDE\\ Boulevard de Pérolles 8\\ Case postale 533\\[1ex] 1701 \textbf{Fribourg}}
	\corpdelettre{~~~\emph{Copie pour information}}
\end{letter}

%{{{1 Lettre spo
\begin{letter}{\adret Service du personnel et d'organisation\\ Rue Joseph-Piller 13\\[1ex] 1700 \textbf{Fribourg}}
	\corpdelettre{}
	\encl{Formulaire de déclaration de refus.}
	\cc{FEDE pour information.}
\end{letter}

%{{{1 Lettre formulaire
\begin{letter}{\adret Service du personnel et d'organisation\\ Rue Joseph-Piller 13\\[1ex] 1700 \textbf{Fribourg}}
\conc{}
\opening{}
	\noindent{%
		\Large%
		\sffamily
		Contribution de soutien en faveur des associations de personnel
		\\[-1ex]\rule{2ex}{.4mm}\\[.6ex]
		\bfseries
		Formulaire de déclaration
	}

	\vspace{3ex}\par\noindent\parbox{5cm}{Nom}: \dotfill
	\vspace{3ex}\par\noindent\parbox{5cm}{Prénom}: \dotfill
	\vspace{3ex}\par\noindent\parbox[t]{5cm}{\No personnel (figure au recto de votre relevé de salaire)}: \dotfill

	\vspace{2ex}\par\noindent{\large\sffamily Mettre une croix dans la case correspondante ci-dessous:}

	\vspace{2ex}\par{\bfseries\raisebox{.5mm}{\fbox{\rule{0mm}{1ex}\rule{1ex}{0mm}} } Déclaration de refus}\\
	Je confirme avoir pris connaissance des dispositions légales relatives à la contribution de soutien.
	Je déclare en toute connaissance de cause refuser le prélèvement de la contribution de soutien et
	prie le centre de paie d'exécuter ma prise de position avec effet pour le mois suivant la date
	d'envoi.

	\vspace{3ex}\par{\bfseries\raisebox{.5mm}{\fbox{\rule{0mm}{1ex}\rule{1ex}{0mm}} } Révocation du refus}\\
	Je confirme avoir pris connaissance des dispositions légales relatives à la contribution de
	soutien. Je déclare en toute connaissance de cause révoquer mon refus du prélèvement de la
	contribution de soutien et prie le centre de paie d'exécuter ma prise de position avec effet pour
	le mois suivant la date d'envoi.

	\vspace{4ex}\par\noindent\parbox{3cm}{Lieu et date}: \dotfill
	\vspace{4ex}\par\noindent\parbox{3cm}{Signature}: \dotfill

\end{letter}

\end{document}
